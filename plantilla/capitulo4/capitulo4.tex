\chapter{Resultados}
\section{Conclusiones}

Los resultados a los que hemos llegado son los siguientes:
\begin{itemize}
\item El tama�o de bloque de memoria que impone el sistema de ficheros se mantiene a nivel f�sico.
\item El montado y desmontado del dispositivo no genera cambios en �l.
\item Analizar las diferencias entre im�genes \verb|iso| tan grandes como las que hemos manejado no es una tarea trivial.
\item Son m�s fr�giles los componentes que acompa�an a la memoria que la propia memoria.
\item Es necesario encontrar un procedimiento m�s eficientes para ``flashear'' memorias que el utilizado en este Proyecto si se quiere continuar con la l�nea de trabajo.
\end{itemize}

\section{L�neas futuras}

Este Proyecto deja varias puertas abiertas a futuras invesgitaciones. 

Ser�a ideal intentar mejorar el script \ref{cod:flash_b4} hasta conseguir los resultados esperados. Uno de los puntos a mejorar es la sobrecarga del dispositivo cuando est� tanto tiempo encendido, trabajando, se podr�a a�adir tiempos de espera, de esta manera posiblemente las memorias no sufran el fallo del controlador o de componentes internos antes de obtener los datos del deterioro de la memoria.

Otro punto que ser�a interesante analizar es el moviento de datos entre sectores de forma m�s eficaz. Esto puede ser evitado usando memorias de menos capacidad, ordenadores con mayor procesado (aunque en este sentido creemos que el fallo de la herramienta \verb|meld| para comparar ficheros tan grandes no depende del ordenador si no del programa en s�), o idear otras maneras de comparar im�genes \verb|.iso| y obtener resultados facilmente interpretables. En nuestro Proyecto hemos desarrollado un peque�o programa en C que compara bit a bit dos ficheros binarios, pero este no lleg� a ser funcional ya que era muy lento y pesado.

Una parte que no nos ha dado tiempo a tratar es la recuperaci�n de datos de una memoria da�ada. M�s all� de las herramientas de software tipo ..., lo interesante es sacar los datos directamente de la memoria f�sica. Ten�amos pensando utilizar un prototipo con una FPGA y un lector de chips de los que se usan para flashear consolas que esta en fase de desarrollo.

Por �ltimo, y siguiendo esta l�nea, se podr�a dise�ar un grabador con una FPGA de esa manera se quitar�a la limitaci�n del controlador y todos los componentes intermedios que pueden fallar antes que la memoria. El incoveniente de este punto es encontrar un dispositivo que aguante mucha carga de trabajo durante mucho tiempo, ya que los lectores de chips que hemos mencionado anteriormente no est�n pensados para estas aplicaciones.
