\section{C�digo}
\subsection{Archivos marcados}
\verb|archivos_marcados_b3.sh|: este script genera ficheros desde un byte hasta el l�mite que se ponga en la variable \verb|valor_final|. Se ha usado para generar datos para llenar la memoria y realizar pruebas.
\begin{lstlisting}[caption=Script que genera datos de varios tama�os.,label=cod:archivos_marcados_b3]
#
# Crea ficheros de todos los tama�os posibles hasta llegar a 'valor_final'
#
valor_final=3907388
((tamfinal=valor_final*1024))

aux=2
cont=0
while [ $aux -le  $tamfinal ]; do

	((aux=aux+aux))
	((cont++))
done

echo "auxiliar" $aux
echo "contado" $cont
rm datos/*
echo -e -n "\x00" > datos/datos0.txt

((aux=cont))

echo "vamos a contar desde 1 a " $aux

for (( a=1 ; a<=aux ; a++ ))
do
	((b=a-1))
	cat datos/datos$b.txt > datos/datos$a.txt
	cat datos/datos$b.txt >> datos/datos$a.txt
	echo $b
	echo $a
done
\end{lstlisting}

\subsection{Flash}
Cada una de las versiones de este script se ha creado con una finalidad concreta tal como describimos en cada uno de ellos.

\verb|flash_b2.sh| : su funci�n es escribir y borrar muchas veces varios datos en una memoria, y monitorizar mediante la escritura en un log la evoluci�n de su capacidad.
\begin{lstlisting}[caption=Script para flashear una memoria con un fichero un gran n�mero de veces.,label=cod:flash_b2]
#
# 
#
dispo="4E06-0AB9"
sdd="sdb"
rm stam.txt	
for (( b=0 ; b<999999999999999999999 ;b++))
do
	let vueltas=$b

for (( a=1 ; a<3 ; a++ ))
do
	echo "numero serie ${b}_${a}"
	echo -e -n "numero serie ${b}_${a}" > datos/${b}_${a}datos.txt
	for (( c=1 ; c<200 ; c++ ))
	do
		echo -e -n "numero serie ${b}_${a}" >> 
datos/${b}_${a}datos.txt
	done
	#cat datos/datos.txt >> datos/${b}_${a}datos.txt
	#echo -e -n "numero serie ${b}_${a}" >> datos/${b}_${a}datos.txt
	dd if=datos/${b}_${a}datos.txt 
of=/media/${dispo}/${b}_${a}datos.txt
	rm datos/*_*datos.txt
done

rm -f /media/${dispo}/*datos.txt

df --block-size=KB | grep ${sdd} >> stam.txt
#cut -c32-52 tam.txt >> stam.txt

frase="vuelta "
echo "$frase $vueltas" >> stam.txt
echo " " >> stam.txt

#sudo umount /media/usb0
#sudo mount /dev/sdc1 /media/usb0

done
\end{lstlisting}


\verb|flash_b3.sh|: esta versi�n se ha hecho con la idea de tener archivos f�cilmente localizables dentro de la imagen de 4GB que generamos y de la cual tenemos que sacar informaci�n y resultados.
\begin{lstlisting}[caption=Script para flashear una memoria con un archivo f�cil de encontrar.,label=cod:flash_b3]
#
# Crea un fichero con un texto reconocible y su numero de serie
#
rm stam.txt
rm diff.txt
for (( b=0 ; b<10 ;b++))
do
	let vueltas=$b

for (( a=1 ; a<3 ; a++ ))
do
	echo "numero serie ${b}_${a}"
	echo -e -n "En un lugar de la Mancha, de cuyo nombre no quiero 
acordarme, no ha mucho tiempo que viv�a un hidalgo de los de lanza en astillero, adarga antigua, roc�n flaco y galgo corredor. Una olla de algo m�s vaca que carnero, salpic�n las m�s 
noches, duelos y quebrantos los s�bados, lentejas los viernes, alg�n palomino de a�adidura los domingos, consum�an las tres partes de su 
hacienda. El resto della conclu�an sayo de velarte, calzas de velludo para las fiestas con sus pantuflos de lo mismo, los d�as de entre 
semana se honraba con su vellori de lo m�s fino. Ten�a en su casa una ama que pasaba de los cuarenta, y una sobrina que no llegaba a los 
veinte, y un mozo de campo y plaza, que as� ensillaba el roc�n como tomaba la podadera. Frisaba la edad de nuestro hidalgo con los 
cincuenta a�os, era de complexi�n recia, seco de carnes, enjuto de rostro; gran madrugador y amigo de la caza. Quieren decir que ten�a el 
sobrenombre de Quijada o Quesada (que en esto hay alguna diferencia en los autores que deste caso escriben), aunque por conjeturas 
veros�miles se deja entender que se llama Quijana; pero esto importa poco a nuestro cuento; basta que en la narraci�n d�l no se salga un 
punto de la verdad. ${b}_${a}" > datos/${b}_${a}datos.txt
	#cat datos/datos.txt >> datos/${b}_${a}datos.txt
	#echo -e -n "numero serie ${b}_${a}" >> datos/${b}_${a}datos.txt
	dd if=datos/${b}_${a}datos.txt of=/media/usb0/${b}_${a}datos.txt
	rm datos/*_*datos.txt
done

dd if=/dev/sdc1 of=copias/${b}.iso

echo "vuelta ${b}" >> diff.txt

if (($b != 0 ))
then
((c=b-1))
cmp -l copias/${b}.iso copias/${c}.iso >> diff.txt
fi

echo "" >> diff.txt

rm -f /media/usb0/*datos.txt

df --block-size=KB | grep sdc1 > tam.txt
cut -c32-52 tam.txt >> stam.txt

frase="vuelta "
echo "$frase $vueltas" >> stam.txt
echo " " >> stam.txt

done
\end{lstlisting}

\verb|flash_b4.sh|: Es la �ltima versi�n del script para flashear una memoria. El �nico cambio importante es que entre la escritura y el borrado de datos, monta y desmonta el dispositivo, con esto nos aseguramos que los datos son escritos correctamente en el disposivo. 
\begin{lstlisting}[caption=�ltima versi�n de script para flashear una memoria.,label=cod:flash_b4]
ruta="/media/cesar/usb3"
sdd="sdb1"
rm stam.txt
for (( b=0 ; b<9999999999999999999999999 ;b++))
do
	let vueltas=$b

for (( a=1 ; a<30 ; a++ ))
do
	echo "numero serie ${b}_${a}"
	echo -e -n "numero serie ${b}_${a}" > datos/${b}_${a}datos.txt
	for (( c=1 ; c<200 ; c++ ))
	do
		echo -e -n "numero serie ${b}_${a}" >> datos/${b}_${a}datos.txt
	done
	#cat datos/datos.txt >> datos/${b}_${a}datos.txt
	#echo -e -n "numero serie ${b}_${a}" >> datos/${b}_${a}datos.txt
	dd if=datos/${b}_${a}datos.txt of=${ruta}/${b}_${a}datos.txt
	rm datos/*_*datos.txt
done
sudo umount ${ruta}
echo "umount"
sudo mount -o gid=1000,uid=1000 /dev/${sdd} ${ruta}
echo "mount"
rm -f ${ruta}/*datos.txt
echo "rm"
sudo umount ${ruta}
echo "umount"
sudo mount -o gid=1000,uid=1000 /dev/${sdd} ${ruta}
echo "mount"

df --block-size=KB | grep ${sdd} >> stam.txt

frase="vuelta "
echo "$frase $vueltas" >> stam.txt
echo " " >> stam.txt

echo "---------------------------------"
echo "---------------------------------"
echo "vuelta"
echo $vueltas

done

exit
\end{lstlisting}

\subsection{Monvimiento de datos}
\verb|mount.sh|: Este script se ha usado para comprobar si en el montado y desmontado, el dispositivo sufre alg�n tipo de cambio. Funciona montado y desmontado el dispositivo varias veces y creado una suma de MD5 que luego se puede comparar.
\begin{lstlisting}[caption=Script para verificar si en el montado y desmontado existe movimiento de datos.,label=cod:mount]
rm md5.txt

for (( a=1 ; a<999999999999999 ; a++ ))
do
	sudo umount /media/usb0
	sudo mount /dev/sdb1 /media/usb0
	echo "$a" >> md5.txt
	dd if=/dev/sdb1 | md5sum >> md5.txt

done
\end{lstlisting}

\subsection{Tama�o de bloque}

\verb|tam_bloque_b1.sh|: este script usa los ficheros generados por el script \ref{cod:archivos_marcados_b3} para generar uno con un tama�o fijo. Se ha usado para llenar una memoria hasta un punto en concreto, de esta manera sab�amos en que posiciones est�bamos grabando el resto de los datos.
\begin{lstlisting}[caption=Script para generar ficheros de un tama�o concreto,label=cod:tam_bloque_b1]
df --block-size=KiB | grep sdb1 > tam.txt
tam_final=$(cut -c56-63 tam.txt)
echo $tam_final
((valor_final=tam_final-6))
((valor_final_bits=valor_final*1024))
echo "valor final" $valor_final_bits
final_binario=$(echo "obase=2; $valor_final_bits"| bc)
echo $final_binario > long.txt
echo "valor en binario" $final_binario
long=${#final_binario}


((long--))

total=0

((exp=long))

for ((a=1;a<=long;a++))
do
bit=$(cut -c$a-$a long.txt)
#echo "a="$a "bit sacado de long.txt" $bit
if((bit==1))
then
	((exponente=2**exp))
#	echo "exponente" $exp
	echo $exp $exponente
	((total=total+exponente))
	
	cat datos/datos$exp.txt >> datos/datosfin.txt
fi
((exp--))
done

echo $total
\end{lstlisting}

\verb|tam_bloque_b2.sh|
\begin{lstlisting}[caption=Script para generar ficheros de un tama�o concreto,label=cod:tam_bloque_b2]
((valor_final=3907388))
#((valor_final=3907388/4))
((valor_final_bits=valor_final*1024))
echo "valor final" $valor_final_bits
final_binario=$(echo "obase=2; $valor_final_bits"| bc)
echo $final_binario > long.txt
echo "valor en binario" $final_binario
long=${#final_binario}


((long--))

total=0

((exp=long))
rm datos/datosfin.txt
for ((a=1;a<=long;a++))
do
bit=$(cut -c$a-$a long.txt)
#echo "a="$a "bit sacado de long.txt" $bit
if((bit==1))
then
	((exponente=2**exp))
#	echo "exponente" $exp
	echo $exp $exponente
	((total=total+exponente))
	
	cat datos/datos$exp.txt >> datos/datosfin.txt
fi
((exp--))
done

echo $total
\end{lstlisting}

\subsection{Tabla de nombres}

\verb|crear_mismo_tam.sh|: la finalidad de este script es copiar el n�mero m�ximo de archivos posibles dentro de una memoria, al prinpio se us� para varias pruebas, pero al final ha sido muy �til a la hora de sacar resultados sobre el funcionamiento de la tabla de nombres.
\begin{lstlisting}[caption=Script para crear muchos ficheros iguales,label=cod:crear_mismo_tam]
valor_final=3907388
((aux=valor_final/4))

echo "vamos a contar desde 1 a " $aux

for (( a=1 ; a<=aux ; a++ ))
do
	echo -e -n "\xAA$a" > /media/B030-FF46/$a.txt
	echo "vuelta $a"
done
\end{lstlisting}
