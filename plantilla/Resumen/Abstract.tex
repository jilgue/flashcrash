\chapter*{Resumen}

En este proyecto hemos profundizado sobre el funcionamiento de las memorias Flash, sobre todo en lo que se refiere al funcionamiento interno, a bajo nivel. Nos hemos centrado en determinar en qu� posiciones de memoria guarda la tabla de nombres, c�mo gestiona los archivos, cu�l es el tama�o que tiene un sector y cu�ntos ciclos de vida �til tiene un sector. Para esta investigaci�n hemos desarrollado una serie de scripts en Bash Shell de Linux que nos han permitido obtener y evaluar los resultados de manera m�s sencilla. Los resultados a los que hemos llegado despu�s de la investigaci�n son que este tipo de tecnolog�a se usa en todo tipo de dispositivos, y tiene mucha progresi�n de futuro. Tras los experimentos, hemos llegado a las siguientes conclusiones: el tama�o de bloque de memoria que impone el sistema de ficheros se mantiene a nivel f�sico, el montado y desmontado del dispositivo no generan cambios en �l y son m�s fr�giles los componentes que acompa�an a la memoria que la propia memoria.

memoria flash, persistencia de datos, l�mite de escritura, tabla de ficheros.



In this project we have deepened over the operation of Flash memories,
especially as it relates to the internal operations at low level. We focused
to determine which memory locations stores table names, how it manages
files, what is the size it is a sector and how many life cycles have a
sector. For this research we have developed a series of scripts in Bash Shell
Linux that have allowed us to obtain and evaluate the results more easily.
The results which have come after the research are that this type
technology is used in all kinds of devices, and has lots of future progression.
After the experiments, we have reached the following conclusions: the block size
memory imposed by the file system remains physically, mounted and
removed from the device does not generate changes, and are more fragile components
accompanying the report that the memory itself.
